% !TeX program = pdflatex
% !TeX TXS-program:compile = txs:///pdflatex/
% !TeX TS-program = pdflatex
% !BIB program = biber
% !TeX TXS-program:bibliography = txs:///biber




%%%%%%%%%%%%%%%%%%%%%%%%%%%%
%%  FUNDAMENTAL SETTINGS  %%
%%%%%%%%%%%%%%%%%%%%%%%%%%%%


\newcommand{\serifbodyfont}{0}
% 0: Body is set in sans-serif font
% 1: Body is set in serif font
\newcommand{\serifheadingfont}{0}
% 0: Headings (``structure'') are set in sans-serif font
% 1: Headings (``structure'') are set in serif font

\newcommand{\showsectionnavigation}{0}
% 0: ``Headline'' is disabled.
% 1: ``Headline'' (at the very top of slides) is included and displays all sections of the presentation,
%      with the current section being highlighted.




%%%%%%%%%%%%%%%%%%%%%%%%%%%%
%%  FUNDAMENTAL PACKAGES  %%
%%%%%%%%%%%%%%%%%%%%%%%%%%%%


\usepackage{ifthen}

\usepackage{calc}

\usepackage{tcolorbox}

\usepackage[LY1, LGR, TS1, T1]{fontenc}
% LGR (Greek) is needed to enable the ``sansserif math'' features.
\usepackage[utf8]{inputenc}

\usepackage{tabularx}

\ifxetex
	\usepackage[protrusion=true, expansion=false]{microtype}
\else
	\usepackage[protrusion=true, expansion=false, kerning=true]{microtype}
\fi
% The microtype package enables so-called hanging punctuation. That is, when punctuation like ":", ".", "-", etc. is found at the beginning or end of a line, it is protruded a little into the page margin. This results in "optical margin alignment": the protrusion makes the margin look straighter.


\definecolor{darkgray}{rgb}{0.4,0.4,0.4}

\usepackage[absolute, overlay]{textpos}
\usepackage{graphicx}
\usepackage{amsmath}
%\usepackage{amsfonts}
%\usepackage{amssymb}
\usepackage{epstopdf}
\DeclareGraphicsRule{.tif}{png}{.png}{`convert #1 `dirname #1`/`basename #1 .tif`.png}

\usepackage[ngerman, american, USenglish]{babel}
% German and US English hyphenation and quotation marks
\selectlanguage{USenglish}
\usepackage[autostyle=true]{csquotes}

\usepackage{multirow}

\usepackage{xcolor}
\usepackage{pgf}%,pgfarrows,pgfnodes,pgfshade}
\usepackage{tikz}
\usepackage{pgfplots}
\usetikzlibrary{mindmap, trees, patterns}

\usepackage{xspace}

% \usepackage{eurosym}

%\usepackage{etoolbox}
%	% Enables manipulating LaTeX commmands via \preto, \appto, \patchcmd, etc.
%	% loaded automatically by beamer
\usepackage{xpatch}
% Enables manipulating LaTeX commmands via \xpatchcmd etc.

\hypersetup{%
	pdfpagemode=FullScreen,
	breaklinks=true,
	colorlinks, linkcolor=, urlcolor=SpotColor, citecolor=%
}
\urlstyle{same}
\newcommand{\email}[1]{\href{mailto:#1}{\nolinkurl{#1}}}

\usepackage{import}	 % To allow for relative paths in nested \input's (\import's)




%%%%%%%%%%%%%%%%%%%%%
%%  FONT SETTINGS  %%
%%%%%%%%%%%%%%%%%%%%%


\subimport{0_0_Preamble/}{Preamble_Fonts_Charter_FiraSans}

\ifnum \serifbodyfont=0
	\setbeamerfont{normal text}{family=\sffamily}
	\renewcommand{\familydefault}{\savesffamily}
	\renewcommand{\mddefault}{\savesfmdseries}
	\renewcommand{\bfdefault}{\savesfbfseries}
\else
	\usefonttheme{serif}
	\setbeamerfont{normal text}{family=\rmfamily}
\fi
\ifnum \serifheadingfont=0
	\setbeamerfont{structure}{family=\sffamily, shape=\upshape, series=\bfseries}
\else
	\setbeamerfont{structure}{family=\rmfamily, shape=\upshape, series=\bfseries}
\fi

% Make frame titles and headings bold
\usefonttheme{structurebold}
\usefonttheme{professionalfonts}

\setbeamerfont{footline}
	{parent=structure, size=\tiny, series=\mdseries}
\setbeamercolor{footline}
	{fg=SpotColor}

%% Use the bm (= boldmath) package for better support of setting math in bold ==>
%% Prevent the "Too many math fonts used" error:
\newcommand{\bmmax}{0}
\newcommand{\hmmax}{0}
\usepackage{bm}
%% <==

\usepackage{fontawesome}

\usepackage{mathtools}
%\mathtoolsset{centercolon}
% This makes the compilation fail in combination with tikz. See
% https://tex.stackexchange.com/questions/89467/why-does-pdftex-hang-on-this-file.
%% Inspired by https://tex.stackexchange.com/questions/251460/how-to-put-symbols-of-equal-size-on-top-of-each-other
\newcommand{\succeqq}{%
	\mathrel{%
		\vcenter{\offinterlineskip
			\ialign{##\cr$\succ$\cr\noalign{\kern 1pt}$=$\cr}%
		}%
	}%
}
\newcommand{\nsucceqq}{\mathrel{\not\succeqq}}
\newcommand*{\coloneqq}{\mathrel{%
		\mathrel{%
			\raisebox{0.15ex}{\scalebox{0.85}{\ensuremath{:}}}\hspace{-0.2pt}%
		}%
		=%
}}

\input{0_0_Preamble/Preamble_Sansserif_math}

% Allow for fine-grained scaling of font sizes
% ==>
\usepackage{relsize}
\renewcommand\RSpercentTolerance{1}
% Enabling slightly reduced font for CAPS with slightly increased letter spacing:
\ifxetex
	\newcommand{\caps}[1]{\textscale{0.96}{\addfontfeature{LetterSpace=3}\MakeUppercase{#1}}}
\else
	\newcommand{\caps}[1]{\textscale{0.96}{\textls[35]{\MakeUppercase{#1}}}}
\fi
% <==




%%%%%%%%%%%%%%%%%%%%%%%%%%%%%%%%%%%%%%%%%%%%%%%%
%%  ADJUSTING THE DESIGN OF THE PRESENTATION  %%
%%%%%%%%%%%%%%%%%%%%%%%%%%%%%%%%%%%%%%%%%%%%%%%%


\setbeamerfont{title}
	{parent=structure, size=\LARGE}
\setbeamerfont{subtitle}
	{parent=structure, size=\Large, series=\mdseries}
\setbeamerfont{frametitle}
	{parent=structure, size=\large}

% For convenience, let the ``paperheight'' of the slides be identical,
% no matter what the aspect ratio of the slides ==>
\makeatletter
% 4:3 aspect ratio:
\@ifclasswith{beamer}{aspectratio=43}{%
	\beamer@paperwidth 12.00cm%
	\beamer@paperheight 9.00cm%
	\setbeamerfont{title}
		{parent=structure, size*={16}{20}}
	\setbeamerfont{subtitle}
		{parent=structure, size*={13.5}{17}}
	\setbeamerfont{frametitle}
		{parent=structure, size*={11}{14}}
}{}
%% 14:9 aspect ratio: Nothing needs to be changed
%\@ifclasswith{beamer}{aspectratio=149}{%
%	\beamer@paperwidth 14.00cm%
%	\beamer@paperheight 9.00cm%
%}{}
%% 16:9 aspect ratio: Nothing needs to be changed
%\@ifclasswith{beamer}{aspectratio=169}{%
%	\beamer@paperwidth 16.00cm%
%	\beamer@paperheight 9.00cm%
%}{}
% 16:10 aspect ratio:
\@ifclasswith{beamer}{aspectratio=1610}{%
	\beamer@paperwidth 14.40cm%
	\beamer@paperheight 9.00cm%
}{}
\makeatother
% <==

% Margins ==>
\newcommand{\margintop}{12.5pt}
\newcommand{\marginleft}{17.5pt}
\newcommand{\marginright}{\marginleft}
\setbeamersize{text margin left=\marginleft}
\setbeamersize{text margin right=\marginright}
\setlength{\textwidth}{\paperwidth-\marginleft-\marginright}
% <==

\setlength{\parskip}{\medskipamount}
% Inserts some space between paragraphs

% Enable hyphenation on Beamer slides:
\usepackage{ragged2e}
\let \raggedright \RaggedRight
\sloppy
\hyphenpenalty=500

\setbeamerfont{alerted text}{series=\bfseries}

%% Lorenz Götte's color scheme:
%\usecolortheme{whale}
%\definecolor{beamer@blendedblue}{rgb}{0.137,0.466,0.741}
%\setbeamercolor{structure}{fg=beamer@blendedblue}
%\definecolor{SpotColor}{rgb}{0.1,0.4,0.7} % Lorenz' Blue
\definecolor{UBonnBlue}   {RGB}{  7,  82, 154}
\definecolor{UBonnYellow} {RGB}{234, 185,  12}
\definecolor{UBonnGray}   {RGB}{144, 144, 133}
\definecolor{darkgray}    {RGB}{102, 102, 102}
\definecolor{darkred}     {RGB}{191,   0,   0}
\definecolor{neutralgreen}{RGB}{ 28, 166,   0}

\colorlet{SpotColor}   {UBonnBlue}
\colorlet{AlertColor}  {darkred}
\colorlet{ExampleColor}{neutralgreen}

\setbeamercolor{structure}{fg = SpotColor}
\setbeamercolor{alerted text}{fg = SpotColor}%{fg=AlertColor}
\setbeamercolor{author in head/foot}{fg = white, bg = SpotColor}
\setbeamercolor{button}{bg = SpotColor, fg = white}
\setbeamercolor{headline}{bg = SpotColor, fg = white}
\setbeamercolor{headlinecover}{bg = white, fg = white}

\newcommand{\highlight}[1]{\textcolor{SpotColor}{#1}}
\newcommand{\heading}[1]{%
	\par%
	{\usebeamerfont{structure}\usebeamercolor[fg]{frametitle}#1}%
	\par%
}
\newcommand{\sigstar}{\highlight{*}}

\newlength{\rulelength}
\setlength{\rulelength}{\paperwidth - \marginleft - \marginright}

% Remove navigation symbols from the slide footer
\beamertemplatenavigationsymbolsempty

% Adjust layout of frame title
% ==>
\setbeamertemplate{frametitle}{%
	\parskip=0pt%
	\vspace{\margintop}\vspace{-5pt}%
	\makebox[0pt][l]{\textcolor{SpotColor}{\rule[-1.2ex]{\rulelength}{0.3pt}}}%
	{\large\usebeamercolor{frametitle}\usebeamerfont{frametitle}\insertframetitle}%
	\par\vspace{-2pt}%
}
% <==

\newlength{\navigationwidth}
\makeatletter
\BeforeBeginEnvironment{frame}{%
	% Add a slide number (and [short]title) to the footer of each slide
	% ==>
	\setbeamertemplate{footline}{\vskip-15pt%
		\hspace{\marginleft}%
		\textcolor{SpotColor}{\rule{\rulelength}{0.3pt}}%
		\vspace{1ex} \linebreak
		\mbox{\hspace{\marginleft}}%
		\strut% to ensure identical height of all footlines
		\textmd{%
			\insertshortauthor%
			\ifdefempty{\beamer@shortdate}{}{~(\insertshortdate)}%
			\ifdefempty{\beamer@shorttitle}{}{: \enquote{\insertshorttitle}}%
		}
		\hfill%
		\strut%
		\textbf{\insertframenumber/\inserttotalframenumber}%
		\hspace{\marginright}%
		\vspace{12pt}%
	}%
	% <==
	% Outline with current section highlighted in the ``headline'' ==>
	\ifnum \showsectionnavigation=0
	\setbeamertemplate{headline}{}
	\else
	\setbeamertemplate{headline}{%
		\hspace{\marginleft}%
		\begin{beamercolorbox}[wd=\linewidth, ht=5pt, dp=2pt]{headline}%
			\insertsectionnavigationhorizontal{\linewidth}{\hspace{0.04\linewidth}}{\hspace{0.04\linewidth}}%
		\end{beamercolorbox}%
		\newline
		\begin{beamercolorbox}[wd=\paperwidth, ht=15pt, dp=0pt]{headlinecover}%
			% Adds a white bar to cover stuff that protrudes from the above color box.
		\end{beamercolorbox}%
	}
	\setlength{\fboxrule}{0.25pt}\setlength{\fboxsep}{7.5pt}%
	\setbeamertemplate{section in head/foot}{%
		\settowidth{\navigationwidth}{\bfseries\insertsectionhead}%
		\fcolorbox{SpotColor!25}{SpotColor}{%
			\makebox[\navigationwidth]{\color{white}\bfseries\insertsectionhead}%
		}%
	}
	\setbeamertemplate{section in head/foot shaded}{%
		\settowidth{\navigationwidth}{\bfseries\insertsectionhead}%
		\fcolorbox{SpotColor}{SpotColor}{%
			\makebox[\navigationwidth]{\color{white}\mdseries\insertsectionhead}%
		}
	}
	\fi
	% <==
}
\makeatother

% See https://tex.stackexchange.com/questions/427257/how-to-remove-footer-for-specific-type-of-slides-frames
% ==>
\makeatletter
\define@key{beamerframe}{standout}[true]{%
	\setbeamertemplate{footline}{\vskip-15pt%
		\hspace{\marginleft}%
		\textcolor{SpotColor}{\rule{\rulelength}{0.3pt}}%
		\vspace{1ex} \linebreak
		\mbox{\hspace{\marginleft}}%
		\strut% to ensure identical height with regular footline
		\hfill%
		\strut%
		\hspace{\marginright}%
		\vspace{12pt}%
	}%
	\setbeamertemplate{headline}{}%
}
\makeatother
% <==

\frenchspacing  % Prevent increased whitespace after periods and colons

\setbeamertemplate{button}{%
	\tikz
	\node[
	inner xsep=3pt,
	inner ysep=2pt,
	draw=structure!100,
	fill=structure!100,
	rounded corners=1.5pt
	]{\raisebox{1.5pt}{\usebeamerfont{structure}\insertbuttontext}};%
}

%\renewcommand{\texteuro}{\fontencoding{TS1}\selectfont\char"BF\fontencoding{T1}\selectfont}
%\newcommand{\euro}{\texteuro}

% Adding framenumbers automatically to the frametitles
\newcommand{\pageinsection}{\number\numexpr\insertpagenumber-\insertsectionstartpage+1}
% From https://tex.stackexchange.com/questions/308343/how-to-create-mini-sections-mini-subsections-and-mini-frames-in-beamer-presenta
% ==>
%\usepackage{etoolbox}  % Automatically loaded by beamer
\makeatletter
\newcount\beamer@sectionstartframe
\beamer@sectionstartframe=1
\apptocmd{\beamer@section}{%
	\addtocontents{nav}{\protect\headcommand{%
			\protect\beamer@sectionframes{\the\beamer@sectionstartframe}{\the\c@framenumber}}}%
}{}{}
\apptocmd{\beamer@section}{%
	\beamer@sectionstartframe=\c@framenumber\advance\beamer@sectionstartframe by1\relax%
}{}{}
\AtEndDocument{%
	\immediate\write\@auxout{\string\@writefile{nav}%
		{\noexpand\headcommand{\noexpand\beamer@sectionframes{\the\beamer@sectionstartframe}{\the\c@framenumber}}}}%
}{}{}
\def\beamer@startframeofsection{1}
\def\beamer@endframeofsection{1}
\def\beamer@sectionframes#1#2{%
	\ifnum\c@framenumber<#1%
	\else%
	\ifnum\c@framenumber>#2%
	\else%
	\gdef\beamer@startframeofsection{#1}%
	\gdef\beamer@endframeofsection{#2}%
	\fi%
	\fi%
}
\newcommand\insertsectionstartframe{\beamer@startframeofsection}
\newcommand\insertsectionendframe{\beamer@endframeofsection}
\makeatother
% <==
\newcommand{\frameinsection}{\number\numexpr\insertframenumber-\insertsectionstartframe+1}
% https://tex.stackexchange.com/questions/228684/two-counters-for-beamer-presentations
%\newcommand{\titleprefix}{\insertsection~{\pageinsection}}
\newcommand{\titleprefix}{\insertsection~{\frameinsection}}
% <==

% Continuation counter for frames with the ``allowframebreaks'' option.
% ==>
% https://github.com/josephwright/beamer/issues/423#issuecomment-456494500:
\makeatletter
\defbeamertemplate*{frametitle continuation}{only if multiple}{%
	\ifnum \numexpr\beamer@endpageofframe+1-\beamer@startpageofframe\relax > 1
		\textmd{(%
			\insertcontinuationcount/%
			\the\numexpr\beamer@endpageofframe+1-\beamer@startpageofframe%
			)%
		}%
	\fi%
	% Correct vertical space at the top of continued frames:
	\ifnum \numexpr\insertcontinuationcount > 1%
		\par\vspace{\medskipamount}
	\fi%
}
\makeatother
% <==




%%%%%%%%%%%%%%%%%%%%%%%%%%%%%%%%%
%%  LAYOUT OF THE TITLE SLIDE  %%
%%%%%%%%%%%%%%%%%%%%%%%%%%%%%%%%%


% Make the title page left-aligned
% ==>
%\setbeamertemplate{title page}[default][left]

% \setbeamerfont{title} and \setbeamerfont{subtitle} already executed above!
\setbeamerfont{author}
	{parent=normal text, size=\large, series=\mdseries, shape=\upshape}
\setbeamercolor{author}
	{parent=normal text}
\setbeamerfont{institute}
	{parent=normal text, size=\small, series=\mdseries, shape=\itshape}
\setbeamercolor{institute}
	{parent=normal text}
\setbeamerfont{date}
	{parent=structure, size=\normalsize, series=\bfseries}
\setbeamercolor{date}
	{parent=normal text}

\makeatletter
\setbeamertemplate{title page}{
	\begin{minipage}[c][6.75cm]{\textwidth}%
		\raggedright%
		\ifx\inserttitlegraphic\@empty\else%
			\inserttitlegraphic%
		\fi
		\vfill
		\begingroup
			\ifx\inserttitle\@empty\else%
				\usebeamerfont{title}\usebeamercolor[fg]{title}\inserttitle\par%
			\fi
			\ifx\insertsubtitle\@empty\else%
				\medskip
				\usebeamerfont{subtitle}\usebeamercolor[fg]{subtitle}\insertsubtitle\par%
			\fi
			\ifx\beamer@shortauthor\@empty\else%
				\vspace{\fill}
				\usebeamerfont{author}\usebeamercolor[fg]{author}\insertauthor\par%
			\fi
			\ifx\insertinstitute\@empty\else%
				\vspace{\fill}
				\usebeamerfont{institute}\usebeamercolor[fg]{institute}\insertinstitute\par%
			\fi
			\ifx\insertdate\@empty\else%
				\vspace{\fill}
				\usebeamerfont{date}\usebeamercolor[fg]{date}\insertdate%
			\fi
		\endgroup
		\vfill
		\vspace{0cm}  % For the \vfill to actually have an effect
	\end{minipage}%
}
\makeatother
% <==

\makeatletter
\renewcommand{\beamer@insttitle}[1]{\highlight{\textsuperscript{\kern.75pt \textit{#1}}}}
\renewcommand{\beamer@instinst}[1]{\beamer@insttitle{#1}\ignorespaces}
\renewcommand{\beamer@andinst}{\\[0.33\baselineskip]}
\makeatother




%%%%%%%%%%%%%%%%%%%%%%%%%%%%%%%%%%%%%%%
%%  LAYOUT OF THE TABLE OF CONTENTS  %%
%%%%%%%%%%%%%%%%%%%%%%%%%%%%%%%%%%%%%%%


\useinnertheme{circles}
\setbeamertemplate{section in toc}{%
	\leavevmode\leftskip=3.3ex%
	\llap{%
		\usebeamerfont*{section number projected}%
		\usebeamercolor{section number projected}%
		\begin{pgfpicture}{-1ex}{0ex}{1ex}{2ex}
			\color{bg}
			\pgfpathcircle{\pgfpoint{0pt}{.75ex}}{1.2ex}
			\pgfusepath{fill}
			\pgftext[base]{\color{fg}\raisebox{0.07ex}{%
					% \addfontfeatures{Numbers={Lining, Monospaced}}%
					\usebeamerfont{structure}%
					\small\inserttocsectionnumber}%
			}
		\end{pgfpicture}\kern1.5ex%
	}%
	\usebeamerfont{normal text}%
	\inserttocsection\par%
}
\setbeamertemplate{subsection in toc}{%
	\leavevmode\leftskip=2em$\bullet$\hskip1em\inserttocsubsection\par%
}

\makeatletter
\patchcmd{\beamer@sectionintoc}
	{\vfill}
	{\medskip}
	{}
	{}
\makeatother




%%%%%%%%%%%%%%
%%  BLOCKS  %%
%%%%%%%%%%%%%%


% Put border around block, exampleblock, and alertblock
% ==>

\usepackage{mdframed}
\usepackage{changepage}

\newlength{\blockinnermargin}
\setlength{\blockinnermargin}{4pt}

\newcommand{\framedblockcolor}{SpotColor}  % This allows us to change the color mid-presentation
\newcommand{\PutBorderAroundBlock}[2]{
	\renewcommand{\framedblockcolor}{#2}%
	\setbeamercolor*{block title}{fg=white, bg=\framedblockcolor}%
	\setbeamercolor*{block body} {fg=black, bg=}%
	\addtobeamertemplate{block begin}{%
		\setbeamercolor*{item}{fg=\framedblockcolor}%
		\begin{mdframed}[linecolor=\framedblockcolor, linewidth=0.35pt, leftmargin=0pt, innerleftmargin=0.375em, innerrightmargin=0.375em, innertopmargin=-0.01ex, innerbottommargin=4pt]%
			\let \insertblocktitleorig \insertblocktitle%
			\renewcommand{\insertblocktitle}{%
				\RaggedRight%
				\leftskip=\blockinnermargin%
				\rightskip=\blockinnermargin%
				\insertblocktitleorig\strut\vspace{-1.5pt}%
					% \strut\vspace{-1.5pt} so that all single-line block titles have identical hight
			}%
	}{%
		\renewcommand{\insertblocktitle}{\insertblocktitleorig}%
		\begin{adjustwidth}{\blockinnermargin}{\blockinnermargin}%
			\RaggedRight%
	}%
	\addtobeamertemplate{block end}{%
		\end{adjustwidth}%
	}{%
		\end{mdframed}%
	}%
}
\PutBorderAroundBlock{}{SpotColor}  % Default color of blocks is SpotColor

\newcommand{\PutBorderAroundBlockExAlert}[2]{%
	\def\temp{#1}%
	\ifx\temp\empty
		\def\blocktype{block }     % Trailing space is crucial!
	\else
		\def\temp{ #1}
		\def\blocktype{block #1 }  % Trailing space is crucial!
	\fi
	\setbeamercolor*{block title\temp}{fg=white, bg=#2}
	\setbeamercolor*{block body\temp} {fg=black, bg=}
	\addtobeamertemplate{\blocktype begin}{%
		\begin{mdframed}[linecolor=#2, linewidth=0.35pt, leftmargin=0pt, innerleftmargin=0.375em, innerrightmargin=0.375em, innertopmargin=-0.01ex, innerbottommargin=4pt]%
			\let \insertblocktitleorig \insertblocktitle%
			\renewcommand{\insertblocktitle}{%
				\RaggedRight%
				\leftskip=\blockinnermargin%
				\rightskip=\blockinnermargin%
				\insertblocktitleorig\strut\vspace{-1.5pt}%
					% \strut\vspace{-1.5pt} so that all single-line block titles have identical hight
			}%
	}{%
		\renewcommand{\insertblocktitle}{\insertblocktitleorig}%
		\begin{adjustwidth}{\blockinnermargin}{\blockinnermargin}%
			\RaggedRight%
	}%
	\addtobeamertemplate{\blocktype end}{%
		\end{adjustwidth}%
	}{%
		\end{mdframed}%
	}%
}
\PutBorderAroundBlockExAlert{example}{ExampleColor}
\PutBorderAroundBlockExAlert{alerted}{AlertColor}

% <==

% Style definition, theorem, lemma, corollary, and proof with out border
% ==>

% Inspired by https://github.com/schnorr/infufrgs/blob/master/slides/beamerthemeInf.sty
\setbeamercolor{theorem title}{fg=SpotColor}
\setbeamerfont{theorem title}{parent=structure}
\setbeamercolor{theorem body}{fg=black}
\setbeamerfont{theorem body}{parent=normal text, shape=\itshape, series=\mdseries}
% Body of theorem, lemma, and corollary are italicized
\AtBeginEnvironment{definition}{%
	\setbeamerfont{theorem body}{parent=normal text, shape=\upshape, series=\mdseries}
	% Body of definition is typeset in upright font
}

\setbeamertemplate{theorem begin}{%
	\usebeamerfont{theorem title}%
	\usebeamercolor[fg]{theorem title}%
	\inserttheoremname%~\inserttheoremnumber%
	\ifx\inserttheoremaddition\empty\else\ (\inserttheoremaddition)\fi%
	%\inserttheorempunctuation\hspace{1ex}%
	\\[\smallskipamount]
	\usebeamerfont{theorem body}%
	\usebeamercolor[fg]{theorem body}%
}
\setbeamertemplate{theorem end}{}

\setbeamercolor{proof title}{fg=SpotColor}
\setbeamerfont {proof title}{parent=structure}
\setbeamercolor{proof body} {fg=black}
\setbeamerfont {proof body} {parent=normal text, shape=\upshape, series=\mdseries}
% Remove period after ``Proof''
% See https://tex.stackexchange.com/questions/31354/how-to-remove-the-in-the-proof-environment
% -->
\makeatletter
\let\@addpunct\@gobble
\makeatother
% <--
\renewcommand{\qedsymbol}{\small\textcolor{SpotColor}{$\blacksquare$}}
\setbeamertemplate{proof begin}{%
	\usebeamerfont{proof title}%
	\usebeamercolor[fg]{theorem title}%
	\insertproofname\\[\smallskipamount]
	\usebeamerfont{proof body}%
	\usebeamercolor[fg]{proof body}%
}
\setbeamertemplate{proof end}{}

% <==




%%%%%%%%%%%%%
%%  LISTS  %%
%%%%%%%%%%%%%


\AtBeginEnvironment{itemize}{%
	\setlength{\partopsep}{-\smallskipamount}%
	\setlength{\labelsep}{0.55em}%
	\setlength{\labelwidth}{\widthof{9.}}%
}
\AtBeginEnvironment{enumerate}{%
	\setlength{\partopsep}{-\smallskipamount}%
	\setlength{\labelsep}{0.5em}%
	\setlength{\labelwidth}{\widthof{9.}}%
}

% Adjust spacing of lists ==>
% Unfortunately, the ``enumitem'' package is incompatible with the ``beamer'' class.
% Hence, we need to adjust left margins etc. manually:
\setlength{\leftmargini}{\widthof{9.}+\labelsep}
\setlength{\leftmarginii}{\widthof{g.}+\labelsep}
\setlength{\leftmarginiii}{\widthof{iii.}+\labelsep}
%\setlength{\rightmargin}{0in}
%\setlength{\itemindent}{0in}
%\usepackage{enumitem}
%\setlist[enumerate, 1]{
%	leftmargin=-\parindent, listparindent=\parindent, labelsep=0.42\parindent, itemsep=\smallskipamount, parsep=0pt
%}
%
%\setlist[itemize, 1]{
%	leftmargin=0pt, listparindent=\parindent, labelsep=0.75em, itemsep=\smallskipamount, parsep=0pt, topsep=1.2ex, label=\usebeamerfont*{itemize item}	\usebeamercolor[fg]{itemize item} \usebeamertemplate{itemize item}, before={\RaggedRight \hyphenpenalty=1000}
%}
%\setlist[itemize, 2]{
%	leftmargin=1.2em, listparindent=\parindent, labelsep=0.6em, itemsep=\smallskipamount, parsep=0pt, topsep=0.6ex, label=\usebeamercolor[fg]{itemize item} --, before={\RaggedRight \hyphenpenalty=1000}
%}
% <==

% ``Medium'' as the bold series:
%\makeatletter
%\def\bfseries@sf{mb}
%\makeatother

%\useinnertheme{circles}

\setbeamertemplate{enumerate items}
[default]
\setbeamertemplate{enumerate subitem}
{\alph{enumii}.}
\setbeamertemplate{enumerate subsubitem}
{\roman{enumiii}.}

\setbeamertemplate{itemize item}
{\raisebox{-0.5pt}{\scalebox{0.95}{\textrm{\textbf{\textbullet}}}}\hspace{-0.06em}}
\setbeamertemplate{itemize subitem}
{\usebeamerfont{normal text}--\hspace{0.15em}}
\setbeamertemplate{itemize subsubitem}
{\raisebox{1.5pt}{\tiny$\blacktriangleright$}\hspace{0.15em}}

\setbeamerfont*{itemize/enumerate body}{size=\normalsize}
\setbeamerfont*{itemize/enumerate subbody}{parent=itemize/enumerate body}
\setbeamerfont*{itemize/enumerate subsubbody}{parent=itemize/enumerate body}

\renewenvironment{quote}{%
	\list{}{\leftmargin\leftmarginii \rightmargin\leftmarginii}%
	\RaggedLeft
	\itshape%
	\item\relax%
}
{\endlist}




%%%%%%%%%%%%%%%%%%%%%%%%%%
%%  FIGURES AND TABLES  %%
%%%%%%%%%%%%%%%%%%%%%%%%%%


\setbeamertemplate{caption}[numbered]
\setbeamertemplate{caption label separator}{. }
\usepackage[justification=centering]{caption}
\ifnum \serifbodyfont=0
	\captionsetup{font={sf, small}}
\else
	\captionsetup{font={rm, small}}
\fi
\ifnum \serifheadingfont=0
	\captionsetup{labelfont={sf, small, bf, color=SpotColor}}
\else
	\captionsetup{labelfont={rm, small, bf, color=SpotColor}}
\fi

\usepackage{tabularx}	% Provides environment tabularx to adjust width of tables
% \usepackage{tabulary}	% For some reason, tabulary doesn't obey the width argument ...
% Emulate the "tabulary" column types:
\newcolumntype{C}{>{\centering\arraybackslash}X}
\newcolumntype{J}{>{\arraybackslash}X}
\newcolumntype{L}{>{\RaggedRight\arraybackslash}X}
\newcolumntype{R}{>{\RaggedLeft\arraybackslash}X}
% \usepackage[flushleft]{threeparttable}	% Provides the tablenotes environment


% Packages for creating better-looking tables
\usepackage{booktabs}
\setlength{\cmidrulewidth}{.35pt}
\setlength{\lightrulewidth}{.35pt}
\setlength{\heavyrulewidth}{.70pt}
\setlength{\abovetopsep}{-8pt}
\addtolength{\aboverulesep}{1pt}	% Make tables a little more spacious
\addtolength{\belowrulesep}{1pt}
% \setlength{\belowbottomsep}{-2pt}
\usepackage{colortbl} % e.g., for colored rules
\arrayrulecolor{SpotColor} % Color all table rules blue

\usepackage{siunitx}[=v2]
% Allows, among others, for alignment of decimal numbers in tables at the decimal point.
\sisetup{
	detect-all,
	round-integer-to-decimal = true,
	group-digits             = true,
	group-minimum-digits     = 5,
	group-separator          = {\kern 1pt},
	table-align-text-pre     = false,
	table-align-text-post    = false,
	input-signs              = + -,
	input-symbols            = {*} {**} {***} \sigstar,
	input-open-uncertainty 	 = ,
	input-close-uncertainty  = ,
	retain-explicit-plus
}
\newcolumntype{T}[1]{@{}S[table-format = #1, table-space-text-pre = {***}, table-space-text-post = {***}]}
% Loading the ``siunitx'' package generates a ``too many math alphabets'' used error.
% See https://tex.stackexchange.com/questions/452682/limit-math-alphabets-created-by-siunitx?noredirect=1&lq=1 and
%https://tex.stackexchange.com/questions/40874/kpfonts-siunitx-and-math-alphabets
% Hence, redefine the (virtually) never used \mathtt command:
\renewcommand{\mathtt}[1]{#1}
% From https://tex.stackexchange.com/questions/213605/siunitx-does-not-detect-semi-bold-font
% ==>
\ExplSyntaxOn\makeatletter
\cs_generate_variant:Nn \tl_if_in:nnTF { noTF }
\cs_set_protected:Npn \__siunitx_detect_font_weight_text: {
	\tl_if_in:noTF { sb bf xb } { \f@series }
	{
		\cs_set:Nn \__siunitx_font_weight: { \boldmath \use:c { \f@series series } }
	}
	{
		\cs_set:Nn \__siunitx_font_weight: { \use:c { \f@series series } }
	}
}
\makeatother\ExplSyntaxOff
% <==

\AtBeginEnvironment{table}{\vspace{-1.2ex}\small}
% Reduces font size in tables slightly and
% reduces excessive white space above tables




%%%%%%%%%%%%%%%%%%%%%%%%%%%%%
%%  BIBLIOGRAPHY SETTINGS  %%
%%%%%%%%%%%%%%%%%%%%%%%%%%%%%


%\usepackage[round]{natbib}

\newlength{\bibindent}
\newlength{\bibitemsep}
\newlength{\bibsep}

\setbeamertemplate{bibliography item}{}
\setbeamercolor{bibliography entry author}  {fg=black}
\setbeamercolor{bibliography entry title}   {fg=black}
\setbeamercolor{bibliography entry location}{fg=black}
\setbeamercolor{bibliography entry note}    {fg=black}

% !TeX program = pdflatex
% !TeX TXS-program:compile = txs:///pdflatex/
% !TeX TS-program = pdflatex
% !BIB program = biber
% !TeX TXS-program:bibliography = txs:///biber




%%%%%%%%%%%%%%%%%%%%%%%%%%%%%%%%%%%%%%%%%%%%%%%%
%%  CITATION COMMANDS AND BIBLIOGRAPHY STYLE  %%
%%%%%%%%%%%%%%%%%%%%%%%%%%%%%%%%%%%%%%%%%%%%%%%%


% AER/JEL/JEP style

\usepackage[
	authordate, backend = biber, natbib = true, doi = only, isbn = false,
	sorting = ynt, sortcites = true,  % sort the in-text citations by year of publication
	compresspages = true, dashed = false,
	backref = true, backrefstyle = three,
	bibencoding = inputenc, citetracker = true,
	noibid  % See https://tex.stackexchange.com/questions/129487/bug-in-the-parencite-command-of-the-biblatex-chicago-package
]{biblatex-chicago}

%%% \urlstyle{same}  % Since biblatex-chicago sets \urlstyle{rm}
%%% % See https://tex.stackexchange.com/questions/134191/line-breaks-of-long-urls-in-biblatex-bibliography:
%%% \setcounter{biburllcpenalty}{10000}
%%% \setcounter{biburlucpenalty}{10000}
%%% \setcounter{biburlnumpenalty}{10000}
%%% \AtBeginDocument{\biburlbigskip = 0mu\relax}  % To prevent excessive whitespace in URLs and DOIs
%%% 
%%% \DeclareDelimFormat{nameyeardelim}{\addcomma\space}
%%% \let \citeorig \cite
%%% \renewcommand{\cite}{\citet}
%%% \renewcommand{\citealp}{\citeorig}
%%% \AtBeginDocument{\setlength{\bibhang}{\baselineskip}}
%%% \DeclareFieldFormat[report]{title}{\mkbibquote{#1}}
%%% \AtBeginDocument{%
%%% 	\iftoggle{cms@comprange}
%%% 		{\DeclareFieldFormat{pages}{\addspace\mbox{\mkcomprange{#1}}}}%
%%% 		{\DeclareFieldFormat{pages}{\addspace\mbox{\mknormrange{#1}}}}%
%%% }%
%%% % Make author/editor name(s) bold
%%% \xpatchbibmacro{author}{\usebibmacro{justauthor}}{\mkbibbold{\usebibmacro{justauthor}}}{}{}
%%% \xpatchbibmacro{author}{\usebibmacro{moreauthor}}{\mkbibbold{\usebibmacro{moreauthor}}}{}{}
%%% 
%%% %\usepackage[backend=biber, natbib=true, bibencoding=inputenc, bibstyle=authoryear, citestyle=authoryear-comp, mincitenames=1, maxcitenames=3, minbibnames=99, maxbibnames=99, uniquename=false, uniquelist=true, backref=true, backrefstyle=three, doi=true, isbn=false, dashed=false, sorting=ynt, sortcites=true, mergedate=true, dateabbrev=false, abbreviate=false, citetracker=true]{biblatex}
%%% % sortcites sorts the in-text citations by year of publication
%%% \DeclareBibliographyAlias{newspaper}{article}
%%% 
%%% \defbibheading{subbibliography}[\refname]{%
%%% 	\section*{#1}%
%%% 	\sectionmark{#1}%
%%% 	\addcontentsline{toc}{section}{#1}%
%%% }
%%% 
%%% \renewcommand{\bibfont}{\sffamily\small}
%%% 	% Reduce font size for the bibliography, make the font sans-serif.
%%% \setlength{\bibindent}{\parindent}
%%% \setlength{\bibitemsep}{0pt}
%%% 
%%% \DefineBibliographyStrings{english}{%
%%% 	andothers = {et~al\adddot},
%%% 	volume = {vol\adddot}
%%% }
%%% 
%%% % Handling of newspaper articles
%%% \DeclareFieldFormat[newspaper]{journaltitle}{%
%%% 	\mkbibemph{#1} \mkbibparens{\thefield{edition}}\addcomma\addspace%
%%% 	\iflanguage{ngerman}{%
%%% 		\addnbspace\thefield{day}\addperiod\addnbspace\mkbibmonth{\thefield{month}}\addspace\thefield{year}%
%%% 	}{% else: default to the USenglish version
%%% 		\mkbibmonth{\thefield{month}}\addnbspace\thefield{day}\addcomma\addspace\thefield{year}%
%%% 	}%
%%% 	\iffieldundef{volume}{\addcolon}{\addcomma}%
%%% }
%%% \DeclareFieldFormat[newspaper]{date}{%
%%% 	\thefield{year}%
%%% }
%%% 
%%% % Adjust formatting of back-references
%%% % ==>
%%% \DefineBibliographyStrings{english}{%
%%% 	backrefpage  = {},	% originally ``cit. on p.''
%%% 	backrefpages = {}	% originally ``cit. on pp.''
%%% }
%%% \DefineBibliographyStrings{ngerman}{%
%%% 	backrefpage  = {},	% originally ``Siehe Seite''
%%% 	backrefpages = {}	% originally ``Siehe Seiten''
%%% }
%%% \renewcommand*{\finentrypunct}{}
%%% \renewbibmacro*{pageref}{%
%%% 	\addperiod
%%% 	\iflistundef{pageref}
%%% 	{}
%%% 	{\printtext[brackets]{%
%%% 			\ifnumgreater{\value{pageref}}{1}
%%% 				{\bibstring{backrefpages}}
%%% 				{\bibstring{backrefpage}}%
%%% 			\printlist[pageref][-\value{listtotal}]{pageref}%
%%% 		}%
%%% 	}%
%%% }
%%% % <==
%%% 
%%% % Move notes to the end of an entry by copying the ``note'' information into ``addendum'': -->
%%% \DeclareSourcemap{
%%% 	\maps[datatype=bibtex]{
%%% 		% Copy values of the Mendeley-created ``annote'' field to the ``note'' field:
%%% 		\map[overwrite]{
%%% 		 	\step[fieldsource=annote]
%%% 		 	\step[fieldset=note, origfieldval, append]
%%% 			\step[fieldset=annote, null]
%%% 		}
%%% 		\map{
%%% 			\step[fieldsource=note, final]
%%% 			\step[fieldset=addendum, origfieldval, final]
%%% 			\step[fieldset=note, null]
%%% 		}
%%% 	}
%%% }
%%% % \DeclareFieldFormat{addendum}{(#1)} % Enclose addendum/note in parentheses.
%%% % <--
%%% 
%%% \AtEveryBibitem{%
%%% 	\ifentrytype{newspaper}%
%%% 		{\clearfield{addendum}}% then
%%% 		{\clearfield{month}}% else
%%% }
%%% 
%%% % Add ``working paper'' as the default value for type of ``techreports'' ==>
%%% % see https://tex.stackexchange.com/questions/212362/how-to-use-declaresourcemap-to-add-default-value-to-a-field
%%% \DeclareSourcemap{
%%% 	\maps[datatype=bibtex]{
%%% 		\map{% Will overwrite fields without the ``overwrite'' option
%%% 			\pertype{techreport}
%%% 			\step[fieldset=type, fieldvalue={Working paper}]
%%% 		}
%%% 	}
%%% }
%%% % <==
%%% 
%%% % Add “doctoral dissertation” as the default value for type of “phdthesis” ==>
%%% % see https://tex.stackexchange.com/questions/212362/how-to-use-declaresourcemap-to-add-default-value-to-a-field
%%% \DeclareSourcemap{
%%% 	\maps[datatype=bibtex]{
%%% 		\map{% Will overwrite fields without the ``overwrite'' option
%%% 			\pertype{phdthesis}
%%% 			\step[fieldset=type, fieldvalue={Doctoral dissertation}]
%%% 		}
%%% 	}
%%% }
%%% % <==
%%% 
%%% % Remove superfluous ``The'' from journal names ==>
%%% \DeclareSourcemap{
%%% 	\maps[datatype=bibtex]{
%%% 		\map{
%%% 			\step[fieldsource=journal, match={\regexp{^The\s}}, replace={}]
%%% 		}
%%% 	}
%%% }
%%% % <==
%%% 
%%% % Replace (Mendeley's) month strings (``jan'', ``feb'', etc.) by full names ==>
%%% \DeclareSourcemap{
%%% 	\maps[datatype=bibtex]{
%%% 		\map{
%%% 			\step[fieldsource=number, match={\regexp{^jan$}}, replace=\regexp{\\mkbibmonth\{1\}}]
%%% 			\step[fieldsource=number, match={\regexp{^feb$}}, replace=\regexp{\\mkbibmonth\{2\}}]
%%% 			\step[fieldsource=number, match={\regexp{^mar$}}, replace=\regexp{\\mkbibmonth\{3\}}]
%%% 			\step[fieldsource=number, match={\regexp{^apr$}}, replace=\regexp{\\mkbibmonth\{4\}}]
%%% 			\step[fieldsource=number, match={\regexp{^may$}}, replace=\regexp{\\mkbibmonth\{5\}}]
%%% 			\step[fieldsource=number, match={\regexp{^jun$}}, replace=\regexp{\\mkbibmonth\{6\}}]
%%% 			\step[fieldsource=number, match={\regexp{^jul$}}, replace=\regexp{\\mkbibmonth\{7\}}]
%%% 			\step[fieldsource=number, match={\regexp{^aug$}}, replace=\regexp{\\mkbibmonth\{8\}}]
%%% 			\step[fieldsource=number, match={\regexp{^sep$}}, replace=\regexp{\\mkbibmonth\{9\}}]
%%% 			\step[fieldsource=number, match={\regexp{^oct$}}, replace=\regexp{\\mkbibmonth\{10\}}]
%%% 			\step[fieldsource=number, match={\regexp{^nov$}}, replace=\regexp{\\mkbibmonth\{11\}}]
%%% 			\step[fieldsource=number, match={\regexp{^dec$}}, replace=\regexp{\\mkbibmonth\{12\}}]
%%% 		}
%%% 	}
%%% }
%%% % <==
%\input{0_0_Preamble/Preamble_BibLaTeX_APA}
\bibliography{Library}

\renewcommand{\bibfont}{\scriptsize}

\renewbibmacro*{pageref}{%
	\addperiod
	\iflistundef{pageref}
	{}
	{\color{UBonnGray}%
		\printtext[brackets]{\caps{PDF}~%
			\ifnumgreater{\value{pageref}}{1}
			{pp.}
			{p.}~%
			\printlist[pageref][-\value{listtotal}]{pageref}%
		}%
	}%
}




%%%%%%%%%%%%%%%%%%%%%%%%%%%%%%%%%
%%  APPENDIX-RELATED SETTINGS  %%
%%%%%%%%%%%%%%%%%%%%%%%%%%%%%%%%%


\usepackage{appendixnumberbeamer}
\AtBeginEnvironment{appendix}{%
	\let \insertframenumberorig \insertframenumber
	\renewcommand{\insertframenumber}{Appendix Slide \insertframenumberorig}
	%\let \inserttotalframenumberorig \inserttotalframenumber
	%\renewcommand{\inserttotalframenumber}{A-\inserttotalframenumberorig}
}




%%%%%%%%%%%%%%%%%%%%%%%%%%
%%  FOR DEBUGGING ONLY  %%
%%%%%%%%%%%%%%%%%%%%%%%%%%


\usepackage{blindtext}
\blindmathtrue

% An auxiliary command to display the current font settings ==>
\makeatletter
\newcommand{\showfont}{%
	\textit{encoding:} \f@encoding{},
	\textit{family:} \f@family{},
	\textit{series:} \f@series{},
	\textit{shape:} \f@shape{},
	\textit{size:} \f@size{}
}
\makeatother
% <==